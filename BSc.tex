\documentclass[11pt,a4paper]{article}

\usepackage{amsmath}
\usepackage[utf8]{inputenc}
\usepackage{graphicx}
\usepackage{biblatex}

\addbibresource{References.bib}

\begin{document}

TITLE\\
AUTHORS\\
DATE

\cleardoublepage{}
\begin{abstract}
  
\end{abstract}
\cleardoublepage{}
\tableofcontents{}
\section{Introduction}

\subsection{Motivation}

\subsection{Top Quarks}

\section{Theory}

\subsection{Standard Model}

\subsubsection{Symmetries and Laws of Conservation} Emmy Noether showed that any
conservation law is associated with a continues symmetry of the Lagrangian. The
conservation laws of classical physics are the result of them being invariant
with respect to their canonically conjugate quantities. The conservation of
Energy, liniear momentum and angular momentum, stems from their invariance in
time, space and angles respectively. This implies that the laws of physics are
independant of the time, the location and the orientation in space. Another
symmetry, that is very important for quantum mechanics, is reflection symmetry,
called parity. A wave function can have positive or negative parity depending on
whether or not it changes sign under parity transformation. For the laws which
are invariant under reflection in space, the parity quantum number $P$ is
conserved, while in relativistic quantum mehcanic, we need to ascribe an
intrinsic parity to particles and antiparticles.\\

Group theory gives us the tools to study these symmetries more elegantly, and
has become particularly useful to describe symmetries of quantum mechanics,
where degenerate eigenstates furnishes irreducible representations of a group.
In studying these groups, we can discover other conserved quantities in the
interactions of the electromagnetic (EM), the weak or the strong force. The
unitary group U(1), leads to conservation of charge in EM and strong
interactions, as well as the conservation of leptum number, at least as far as
we know, in all interactions. Certain particles behave practically identically
with respect to the strong or the weak interactions, these properties are
studied through irreducible representations of the group SU(3), and are
characterized by strong and weak isospin, which are also conserved.

\subsection{Four-momentum}

\section{Apparatus}

\subsection{LHC}

\subsection{ATLAS}

\section{Data Analysis}

\subsection{The Code}

\subsubsection{ROOT}

\subsubsection{Optimization??}

\subsection{Results}

\section{Conclusion}

\printbibliography

\end{document}
