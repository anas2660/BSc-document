\documentclass[11pt,a4paper]{article}

\usepackage{amsmath}
\usepackage[utf8]{inputenc}
\usepackage{graphicx}
\usepackage{biblatex}

\addbibresource{References.bib}

\begin{document}

TITLE\\
AUTHORS\\
DATE

\cleardoublepage{}
\begin{abstract}
  
\end{abstract}
\cleardoublepage{}
\tableofcontents{}
\section{Introduction}

\subsection{Motivation}

\subsection{Top Quarks}

\section{Theory}

\subsection{Standard Model}
The Standard Model is a physical theory that unifies
three of the four fundamental forces, the electromagnetic force, the strong
force and the weak force. Attempts have been made to unify the standard model
with the final fundamental force, the gravitational force as described by
general relativity. These theories include loop quantum gravity and string
theory, however no experimental evidence has been gathered for either theory and
testing these theories is very difficult with our current measuring
equipment(?).\\
%"with current particle accelerators" måske?

The Standard Model contains all known elementary particles, they can be split
into fermions and bosons. Fermions can further be split up into Leptons and
Quarks. Fermions have spin $\frac{1}{2}$ and are grouped into three different
generations or families based on their masses, they obey the pauli-exclusion
principle which means that two identical fermions can't have the same quantum
numbers in a system. Each fermion has an antifermion with same mass but opposite
charge, colour and third component of weak isospin. \\
%that are split into two distinct categories, called fermions and bosons "is
%further split", eller bare "is split"

Bosons include the photon, gluon, W- and Z-boson. Which respectively are the
force carriers of the electromagnetic force, the strong force and the weak
force. Bosons unlike fermions don't obey the Pauli exclusion principle.
% måske lidt mere om kvarker (colour confinement) og kort beskrivelse til force
% carriers(?) (altså deres styrke, men ved ikke om det er vigtigt.
%jeg synes at det er en god intro og leder godt ind i afsnittet under om
%symmetries and laws of conservation.

\subsubsection{Symmetries and Laws of Conservation} Emmy Noether showed that any
conservation law is associated with a continues symmetry of the Lagrangian. The
conservation laws of classical physics are the result of them being invariant
with respect to their canonically conjugate quantities. The conservation of
energy, linear momentum and angular momentum, stems from their invariance in
time, space and angles respectively. This implies that the laws of physics are
independent of the time, the location and the orientation in space.\\

Another symmetry, that is very important for quantum mechanics, is reflection
symmetry, called parity. A wave function can have positive or negative parity
depending on whether or not it changes sign under parity transformation. For the
laws which are invariant under reflection in space, the parity quantum number
$P$ is conserved, while in relativistic quantum mechanic, we need to ascribe an
intrinsic parity to particles and antiparticles.\\

Group theory gives us the tools to study these symmetries more elegantly, and
has become particularly useful to describe symmetries of quantum mechanics,
where degenerate eigenstates furnishes irreducible representations of a group.
In studying these groups, we can discover other conserved quantities in the
interactions of the electromagnetic (EM), the weak or the strong force. \\

The unitary group U(1), leads to conservation of charge in EM and strong
interactions, as well as the conservation of lepton number, at least as far as
we know, in all interactions. Certain particles behave practically identically
with respect to the strong or the weak interactions, these properties are
studied through irreducible representations of the group SU(3), and are
characterized by strong and weak isospin, which are also conserved.

\subsection{Four-momentum}

\section{Apparatus}

\subsection{LHC}

\subsection{ATLAS}
% https://atlas.web.cern.ch/Atlas/TP/NEW/HTML/tp9new/tp9.html
% https://atlas.cern/discover/detector
% https://en.wikipedia.org/wiki/ATLAS_experiment#ATLAS_detector
% Gamle hjemmeside: https://web.archive.org/web/20110614103207/http://atlas.ch/detector.html

% potential figure https://commons.wikimedia.org/wiki/File:ATLAS_Drawing.jpg

The ATLAS detector consists of multiple primary components, each with their own subcomponents or subsections.

\subsubsection{The Inner Detector}
% https://atlas.web.cern.ch/Atlas/TP/NEW/HTML/tp9new/node10.html#SECTION00433000000000000000
% https://atlas.cern/discover/detector/inner-detector
% https://en.wikipedia.org/wiki/ATLAS_experiment#Inner_Detector
The purpose of this component is to measure charge and momentum, including it's
direction, of the detected particle. It consists of three subcomponents.
\paragraph{Pixel Detector}
\paragraph{Semiconductor Tracker}
\paragraph{Transition Radiation Tracker.}

\subsubsection{Calorimeter}
% https://atlas.web.cern.ch/Atlas/TP/NEW/HTML/tp9new/node9.html#SECTION00432000000000000000
% https://atlas.cern/discover/detector/calorimeter
% https://en.wikipedia.org/wiki/ATLAS_experiment#Calorimeters
This component will stop particles moving through while measuring the
energy they lose by being stopped. % måske lidt kringlet og er passivt
The Calorimeter can not stop muons and neutrinos.

% 2 subcomponents
\paragraph{Liquid Argon Calorimeter}
\paragraph{Tile Hadronic Calorimeter}

\subsubsection{Muon Spectrometer}
% https://atlas.web.cern.ch/Atlas/TP/NEW/HTML/tp9new/node11.html#SECTION00434000000000000000
% https://atlas.cern/discover/detector/muon-spectrometer
% https://en.wikipedia.org/wiki/ATLAS_experiment#Muon_Spectrometer
This detects the muons, which have passed through the calorimeter and inner
detector, and measures their momenta.

\paragraph{Thin Gap Chambers}
\paragraph{Resistive Plate Chambers}
\paragraph{Monitored Drift Tubes}
\paragraph{Cathode Strip Chambers}

\subsubsection{Magnet System}
% https://atlas.web.cern.ch/Atlas/TP/NEW/HTML/tp9new/node8.html#SECTION00431000000000000000
% https://atlas.cern/discover/detector/magnet-system
% https://en.wikipedia.org/wiki/ATLAS_experiment#Magnet_system
This system bends the path of the particles, such that their track stays within
the confines of the detector.

\paragraph{Central Solenoid Magnet}
\paragraph{Barrel Toroid}
\paragraph{End-cap Toroids}

\subsubsection{Data Collection}
% https://atlas.web.cern.ch/Atlas/TP/NEW/HTML/tp9new/node12.html
% https://en.wikipedia.org/wiki/ATLAS_experiment#Data_systems
% https://atlas.cern/discover/detector/trigger-daq
The detector generates 60 terabytes of data per second from the 1.7 billion
collisions taking place in the detector in that time frame. Therefore, The ATLAS
Detector has a hardware trigger system which helps selectively save only data in
which a particle has been detected.

% HLT = CPU farm but not trigger? data compiler?

\section{Data Analysis}

\subsection{The Code}

\subsubsection{ROOT}

\subsubsection{Optimization??}

\subsection{Results}

\section{Conclusion}

\printbibliography

\end{document}
